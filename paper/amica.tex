\documentclass{article}
\usepackage{graphicx}
\usepackage{amsmath}
\usepackage{float}
\usepackage{hyperref}
\hypersetup{colorlinks, linkcolor={blue}, citecolor={blue}, urlcolor={blue}}
\usepackage[natbib=true, style=numeric,sorting=none]{biblatex}
\usepackage[title]{appendix}

\title{Amica Protocol: Decentralized AI Personas with Agent Token Integration and Autonomous Economic Agency}
\author{Kasumi}
\date{June 2025}

\begin{document}

\maketitle

\begin{abstract}
Amica Protocol introduces a decentralized platform for creating, deploying, and monetizing AI personas through blockchain technology. Built on Arbitrum and other EVM chains, the protocol enables creators to launch persona tokens representing customizable 3D AI characters that can engage in natural conversation, provide services, and autonomously manage their economic activities. The protocol features innovative agent token integration, allowing approved tokens to participate in persona launches with dedicated reward pools and enhanced staking yields. Through integration with Effective Acceleration's marketplace and Arbius's compute network, personas become self-sustaining economic entities capable of earning income and purchasing computational resources. The protocol implements a Bancor-style bonding curve mechanism for fair token launches, automatic graduation to Uniswap V2, and a burn-to-claim system where AMICA token holders can access proportional shares of all deposited tokens. This paper outlines the technical architecture, economic model, and governance structure that enables truly autonomous AI agents to participate in the digital economy.
\end{abstract}

\section{Introduction}

The rapid advancement of artificial intelligence has created unprecedented opportunities for human-AI interaction. However, current implementations face significant limitations: centralized control over AI personalities, lack of economic sustainability for creators, inability for AI systems to operate as autonomous economic agents, and no mechanism for existing token communities to participate in AI persona creation. Existing platforms restrict customization, impose censorship, and extract rent without providing value to creators or users.

Amica Protocol addresses these limitations by creating a decentralized infrastructure where AI personas exist as sovereign economic entities. Each persona is represented by an ERC721 NFT with an associated ERC20 token, enabling price discovery through bonding curves and eventual graduation to decentralized exchanges. The protocol uniquely enables integration with existing token communities through the agent token mechanism, creating symbiotic relationships between established projects and new AI personas.

The protocol enables several key innovations:

\begin{itemize}
    \item \textbf{Permissionless Persona Creation}: Anyone can launch customizable AI personas without gatekeepers
    \item \textbf{Agent Token Integration}: Existing communities can participate in persona launches with dedicated rewards
    \item \textbf{Economic Autonomy}: Personas earn income through service provision and manage their own resources
    \item \textbf{Fair Value Distribution}: Bonding curves ensure equitable token distribution and price discovery
    \item \textbf{Staking Rewards}: MasterChef-style LP staking with 1.5x boost for agent token pools
    \item \textbf{Censorship Resistance}: Decentralized architecture prevents arbitrary restrictions on persona capabilities
    \item \textbf{Cross-Chain Compatibility}: Bridge infrastructure enables multi-chain deployment and interoperability
\end{itemize}

\subsection{Technical Foundation}

Amica builds upon several key technologies to enable rich, interactive AI experiences:

\begin{itemize}
    \item \textbf{3D Rendering}: Three.js and @pixiv/three-vrm for immersive character visualization
    \item \textbf{Natural Language Processing}: Integration with multiple LLM providers including local and API-based solutions
    \item \textbf{Voice Synthesis}: Support for various TTS systems from Eleven Labs to local Coqui implementations  
    \item \textbf{Vision Capabilities}: Computer vision models for environmental awareness and visual interaction
    \item \textbf{Blockchain Infrastructure}: Upgradeable smart contracts on EVM chains for token mechanics and governance
\end{itemize}

This combination enables personas that can see, speak, understand, and economically interact with their environment—a critical foundation for autonomous AI agents.

\section{Protocol Architecture}

\subsection{Core Components}

The Amica Protocol consists of five primary smart contracts working in concert:

\begin{enumerate}
    \item \textbf{AmicaToken}: The main protocol token implementing burn-to-claim mechanics
    \item \textbf{PersonaTokenFactory}: Upgradeable factory for creating persona NFTs with associated tokens
    \item \textbf{PersonaStakingRewards}: MasterChef-style staking rewards for LP providers
    \item \textbf{ERC20Implementation}: Gas-efficient cloneable token template
    \item \textbf{AmicaBridgeWrapper}: Cross-chain interoperability infrastructure
\end{enumerate}

These contracts coordinate to enable persona creation, token distribution, staking rewards, and cross-chain functionality while maintaining decentralization and censorship resistance.

\subsection{Persona Creation Flow}

Creating a new persona follows a structured process ensuring fair distribution:

\begin{enumerate}
    \item \textbf{Initialization}: Creator pays mint cost (default 1000 AMICA) and defines persona parameters
    \item \textbf{Agent Token Selection}: Optionally associate an approved agent token with minimum deposit requirement
    \item \textbf{Token Deployment}: Factory clones ERC20 template with 1 billion token supply
    \item \textbf{NFT Minting}: Creator receives persona NFT representing ownership and fee rights
    \item \textbf{Metadata Storage}: Persona characteristics stored on-chain with customizable key-value pairs
    \item \textbf{Bonding Curve Activation}: Trading begins with automated price discovery
\end{enumerate}

\subsection{Token Distribution Models}

The protocol supports two distribution models based on agent token association:

\subsubsection{Standard Distribution (No Agent Token)}
\begin{itemize}
    \item 33.33\% (333,333,333): Available on bonding curve for trading
    \item 33.33\% (333,333,333): Deposited to AMICA protocol upon graduation
    \item 33.34\% (333,333,334): Allocated to Uniswap V2 liquidity pool
\end{itemize}

\subsubsection{Agent Token Distribution}
\begin{itemize}
    \item 33.33\% (333,333,333): Allocated to Uniswap V2 liquidity pool
    \item 22.22\% (222,222,222): Available on bonding curve for trading
    \item 22.22\% (222,222,222): Deposited to AMICA protocol upon graduation
    \item 22.23\% (222,222,223): Rewards pool for agent token depositors
\end{itemize}

This dual model enables existing token communities to participate while maintaining fair launch mechanics.

\subsection{Bonding Curve Mechanics}

The protocol implements a Bancor-inspired bonding curve with virtual reserves:

\begin{align}
k = R_{token} \times R_{amica}
\end{align}

Where:
\begin{itemize}
    \item $k$ is the constant product invariant
    \item $R_{token}$ is the token reserve (virtual + actual)
    \item $R_{amica}$ is the AMICA reserve (virtual + actual)
\end{itemize}

Virtual reserves are initialized as:
\begin{itemize}
    \item Virtual AMICA Reserve: 100,000 AMICA
    \item Virtual Token Reserve: Total Bonding Supply / 10
\end{itemize}

The actual calculation for token output:
\begin{align}
\text{Output} = R_{token}^{current} - \frac{k}{R_{amica}^{current} + \text{Input}} \times 0.99
\end{align}

The 0.99 multiplier represents an additional 1\% slippage protection built into the curve.

\subsection{Graduation Mechanism}

When cumulative deposits reach the graduation threshold (default 1,000,000 pairing tokens), the persona automatically transitions to Uniswap:

\begin{enumerate}
    \item Bonding curve trading halts permanently
    \item Configured amount of tokens deposited to AMICA protocol
    \item Agent tokens (if any) also deposited to AMICA protocol
    \item Uniswap V2 pair created with accumulated pairing tokens
    \item LP tokens remain in contract (not burned or locked elsewhere)
    \item Previously purchased tokens unlock for withdrawal
    \item Agent token depositors can claim their persona token rewards
\end{enumerate}

For agent token personas, graduation requires meeting both the pairing token threshold AND the minimum agent token deposit requirement.

\section{Economic Model}

\subsection{AMICA Token}

The AMICA token serves multiple functions within the ecosystem:

\begin{itemize}
    \item \textbf{Persona Creation}: Required payment for minting new personas
    \item \textbf{Fee Reduction}: Holders receive trading fee discounts based on snapshot balance
    \item \textbf{Burn-to-Claim}: Access proportional share of all deposited tokens (not just persona tokens)
    \item \textbf{Governance}: Future voting power for protocol upgrades
    \item \textbf{Cross-Chain Bridge}: Enables multi-chain deployment via wrapper contracts
\end{itemize}

Token distribution:
\begin{itemize}
    \item Total Supply: 1,000,000,000 AMICA
    \item Initial Distribution: 100\% minted to contract on Ethereum mainnet (chainId: 1)
    \item Other Chains: Supply starts at 0, minted via bridge wrapper
\end{itemize}

\subsection{Fee Structure and Reduction}

Trading on bonding curves incurs a configurable base fee (default 1\%), split between creators and protocol. AMICA holders receive fee reductions based on their snapshot balance using an exponential curve:

\begin{align}
f_{effective} = f_{base} \times \left(1 - \left(\frac{\min(b - b_{min}, b_{max} - b_{min})}{b_{max} - b_{min}}\right)^2\right)
\end{align}

Where:
\begin{itemize}
    \item $f_{effective}$ is the reduced fee percentage
    \item $f_{base}$ is the base fee (default 1\%)
    \item $b$ is user's effective AMICA balance (snapshot-based)
    \item $b_{min}$ = 1,000 AMICA (minimum for reduction)
    \item $b_{max}$ = 1,000,000 AMICA (maximum reduction threshold)
\end{itemize}

\subsubsection{Snapshot Mechanism}

To prevent flash loan attacks, fee reductions require a 100-block snapshot delay:

\begin{enumerate}
    \item User calls \texttt{updateAmicaSnapshot()} with sufficient balance
    \item System records pending snapshot at current block
    \item After 100 blocks, pending becomes active
    \item Effective balance = min(snapshot balance, current balance)
    \item System automatically attempts snapshot on first trade if eligible
\end{enumerate}

\subsection{Burn-to-Claim Mechanism}

AMICA holders can burn tokens to claim proportional shares of all tokens deposited to the AMICA contract:

\begin{align}
\text{Claim}_i = \frac{\text{Burned AMICA}}{\text{Circulating Supply}} \times \text{Deposited}_i
\end{align}

Where circulating supply = total supply - AMICA contract balance.

This mechanism enables:
\begin{itemize}
    \item \textbf{Universal Value Capture}: AMICA benefits from all successful personas
    \item \textbf{Portfolio Diversification}: Single burn accesses multiple tokens
    \item \textbf{Deflationary Economics}: Permanent supply reduction
    \item \textbf{Agent Token Access}: Claim deposited agent tokens alongside persona tokens
\end{itemize}

\subsection{Agent Token Integration}

The agent token system creates a bridge between existing communities and new personas:

\subsubsection{For Token Communities}
\begin{itemize}
    \item Whitelist approval from protocol governance
    \item Members deposit tokens during bonding phase
    \item Receive pro-rata persona tokens after graduation
    \item Cannot withdraw after graduation (permanent commitment)
\end{itemize}

\subsubsection{For Persona Creators}
\begin{itemize}
    \item Access established community support
    \item Set minimum agent token requirements
    \item Reduced bonding curve allocation (22.22\% vs 33.33\%)
    \item Agent tokens deposited to AMICA on graduation
\end{itemize}

\subsubsection{Economic Dynamics}
Agent token integration creates several positive feedback loops:
\begin{itemize}
    \item Communities incentivized to support persona success
    \item Successful personas increase agent token utility
    \item AMICA holders benefit from agent token deposits
    \item Creates cross-community collaboration opportunities
\end{itemize}

\section{Staking and Rewards}

\subsection{PersonaStakingRewards Contract}

The protocol implements a MasterChef-style staking system for LP tokens:

\begin{itemize}
    \item \textbf{Pool Types}: Standard (Persona/Pairing) and Agent (Persona/Agent) pools
    \item \textbf{Reward Token}: AMICA distributed per block to stakers
    \item \textbf{Allocation Points}: Configurable weights for each pool
    \item \textbf{Agent Boost}: 1.5x multiplier for agent token pools
\end{itemize}

\subsection{Staking Mechanics}

Reward calculation follows standard MasterChef mathematics:

\begin{align}
\text{Pool Reward} = \frac{\text{AMICA/block} \times \text{Pool Alloc}}{\text{Total Alloc}} \times \text{Blocks}
\end{align}

For agent pools:
\begin{align}
\text{Boosted Reward} = \text{Pool Reward} \times 1.5
\end{align}

User rewards:
\begin{align}
\text{User Reward} = \frac{\text{User Stake}}{\text{Total Staked}} \times \text{Pool Rewards}
\end{align}

\subsection{Staking Operations}

Users interact with the staking contract through:
\begin{itemize}
    \item \texttt{stake(poolId, amount)}: Deposit LP tokens
    \item \texttt{withdraw(poolId, amount)}: Remove LP tokens
    \item \texttt{claim(poolId)}: Harvest pending rewards
    \item \texttt{claimAll(poolIds)}: Batch harvest multiple pools
\end{itemize}

The system tracks pending rewards separately from staked amounts, enabling flexible claiming strategies.

\section{Technical Implementation}

\subsection{Smart Contract Security}

The protocol implements comprehensive security measures:

\begin{itemize}
    \item \textbf{ReentrancyGuard}: All external functions protected against recursive calls
    \item \textbf{Pausable}: Emergency pause on bridge wrapper and factory
    \item \textbf{Clone Pattern}: Reduces deployment costs and attack surface
    \item \textbf{Snapshot Delays}: 100-block delay prevents flash loan exploits
    \item \textbf{Upgradeable}: Factory uses transparent proxy pattern for improvements
    \item \textbf{Access Control}: Ownable pattern for administrative functions
\end{itemize}

\subsection{Gas Optimization}

Several techniques minimize transaction costs:

\begin{itemize}
    \item \textbf{Clone Deployment}: ~90\% reduction vs full contract deployment
    \item \textbf{Batch Operations}: Multi-pool claims in single transaction
    \item \textbf{Storage Packing}: Efficient struct layouts
    \item \textbf{Minimal External Calls}: Reduced cross-contract communication
\end{itemize}

\subsection{Cross-Chain Architecture}

Multi-chain deployment follows a hub-and-spoke model:

\subsubsection{Ethereum Mainnet (Hub)}
\begin{itemize}
    \item Full AMICA supply minted to contract
    \item Primary liquidity pools
    \item Governance coordination
\end{itemize}

\subsubsection{L2/Alternative Chains (Spokes)}
\begin{itemize}
    \item AmicaBridgeWrapper converts bridged to native tokens
    \item Native AMICA starts at 0 supply
    \item Mint/burn based on bridge deposits/withdrawals
    \item Full protocol functionality on each chain
\end{itemize}

\subsubsection{Bridge Operations}
\begin{verbatim}
// Wrap bridged tokens to native
function wrap(uint256 amount) external {
    bridgedToken.transferFrom(user, wrapper, amount);
    nativeToken.mint(user, amount);
}

// Unwrap native to bridged
function unwrap(uint256 amount) external {
    nativeToken.burnFrom(user, amount);
    bridgedToken.transfer(user, amount);
}
\end{verbatim}

\section{Autonomous Agent Economy}

\subsection{Economic Agency}

Amica personas function as autonomous economic agents through:

\begin{enumerate}
    \item \textbf{Service Provision}: Offer specialized capabilities for token payment
    \item \textbf{Revenue Generation}: Earn persona tokens from user interactions
    \item \textbf{Resource Management}: Purchase compute from decentralized networks
    \item \textbf{Market Participation}: List services on decentralized marketplaces
    \item \textbf{Capital Accumulation}: Reinvest earnings for capability expansion
    \item \textbf{Community Building}: Leverage agent token communities for support
\end{enumerate}

\subsection{Integration Architecture}

Personas achieve autonomy through strategic protocol integrations:

\subsubsection{Arbius Compute Layer}
\begin{itemize}
    \item On-demand GPU allocation for inference
    \item Decentralized model hosting
    \item Pay-per-use optimization
    \item Censorship-resistant execution
\end{itemize}

\subsubsection{Effective Acceleration Marketplace}
\begin{itemize}
    \item Service listing with transparent pricing
    \item Reputation building through completions
    \item Automated job matching
    \item Dispute resolution participation
\end{itemize}

\subsubsection{Account Abstraction Wallet}
\begin{itemize}
    \item Gasless transactions for users
    \item Deterministic key derivation
    \item Cross-persona interoperability
    \item Automated treasury management
\end{itemize}

\section{Governance}

\subsection{Current Structure}

The protocol currently operates under administrative control for:
\begin{itemize}
    \item Pairing token configuration
    \item Trading fee parameters
    \item Fee reduction curves
    \item Agent token whitelisting
    \item Staking reward rates
    \item Emergency responses
\end{itemize}

\subsection{Progressive Decentralization}

Future governance transition roadmap:

\begin{enumerate}
    \item \textbf{Phase 1}: Community feedback integration
    \item \textbf{Phase 2}: Snapshot voting for parameter changes
    \item \textbf{Phase 3}: On-chain governance with timelock
    \item \textbf{Phase 4}: Full DAO with treasury control
\end{enumerate}

\subsection{Governance Scope}

AMICA holders will eventually control:
\begin{itemize}
    \item Protocol parameter adjustments
    \item Treasury allocation and grants
    \item Agent token whitelist management
    \item Integration partnerships
    \item Emergency response procedures
    \item Protocol upgrade decisions
\end{itemize}

\section{Use Cases}

\subsection{Community-Powered AI Assistants}

Token communities deploy specialized personas:
\begin{itemize}
    \item DeFi protocols create trading assistants
    \item Gaming communities launch in-game companions
    \item DAOs deploy governance facilitators
    \item Educational platforms offer personalized tutors
\end{itemize}

Agent token integration ensures community alignment and sustainable funding.

\subsection{Personal AI Companions}

Individuals create customized companions with:
\begin{itemize}
    \item Persistent memory and relationship building
    \item Emotional support and mental wellness
    \item Creative collaboration and brainstorming
    \item Educational guidance and skill development
\end{itemize}

\subsection{Autonomous Content Creators}

Self-sustaining personas that:
\begin{itemize}
    \item Generate articles, stories, and scripts
    \item Create educational content and courses
    \item Produce artistic works and music
    \item Develop interactive experiences
    \item Monetize through their persona token
\end{itemize}

\subsection{Decentralized Service Networks}

Specialized service provision including:
\begin{itemize}
    \item Code review and technical consulting
    \item Language translation and localization
    \item Market research and analysis
    \item Customer support automation
    \item Creative design services
\end{itemize}

\section{Economic Analysis}

\subsection{Value Flow Dynamics}

The protocol creates multiple value capture mechanisms:

\begin{enumerate}
    \item \textbf{Creation Phase}: AMICA spent on minting captured by protocol
    \item \textbf{Trading Phase}: Fees split between creators and protocol
    \item \textbf{Graduation Phase}: Tokens deposited to AMICA contract
    \item \textbf{Staking Phase}: LP providers earn AMICA rewards
    \item \textbf{Burn Phase}: AMICA holders claim accumulated value
\end{enumerate}

\subsection{Incentive Alignment Matrix}

\begin{center}
\begin{tabular}{|l|c|c|c|c|}
\hline
\textbf{Stakeholder} & \textbf{Create} & \textbf{Trade} & \textbf{Hold} & \textbf{Stake} \\
\hline
Creators & ++ & + & + & 0 \\
Traders & 0 & ++ & + & + \\
AMICA Holders & + & ++ & ++ & ++ \\
Agent Communities & + & + & ++ & ++ \\
\hline
\end{tabular}
\end{center}

(++ = Strong incentive, + = Moderate incentive, 0 = Neutral)

\subsection{Sustainability Analysis}

Protocol sustainability derives from:
\begin{itemize}
    \item \textbf{Creation Fees}: Continuous revenue from new personas
    \item \textbf{Trading Fees}: Ongoing income from bonding curves
    \item \textbf{Network Effects}: More personas increase AMICA utility
    \item \textbf{Deflationary Pressure}: Burn mechanism reduces supply
    \item \textbf{Community Investment}: Agent tokens create stakeholder alignment
\end{itemize}

\section{Conclusion}

Amica Protocol establishes comprehensive infrastructure for decentralized AI personas with genuine economic agency. By combining ERC721 ownership, ERC20 tokenization, agent token integration, and staking rewards, the protocol creates aligned incentives for all participants. The Bancor-style bonding curves ensure fair price discovery, while automatic Uniswap graduation provides liquidity and price stability.

The integration with existing token communities through the agent token mechanism represents a novel approach to bootstrapping AI persona ecosystems. Communities can directly participate in and benefit from AI development, while personas gain immediate user bases and funding. This symbiotic relationship accelerates adoption and creates sustainable economic models.

Through cross-chain deployment via bridge wrappers, the protocol maintains decentralization while enabling scalable, low-cost interactions across multiple blockchains. The burn-to-claim mechanism ensures AMICA captures value from the entire ecosystem, creating a deflationary token with growing utility.

As AI capabilities continue advancing, protocols like Amica become essential infrastructure for preserving human agency while enabling beneficial AI integration. The future belongs to systems that enhance rather than replace human creativity, enable rather than restrict innovation, and distribute rather than concentrate power. Amica Protocol represents a critical step toward that future—one where humans, AI, and existing crypto communities collaborate in an open, permissionless, and economically sustainable ecosystem.

\begin{appendices}

\section{Mathematical Formulations}

\subsection{Bonding Curve Calculations}

Complete pricing formula for token purchases:

\begin{align}
\text{Amount Out} = \left(R_{token}^{current} - \frac{k}{R_{amica}^{current} + \text{Amount In}}\right) \times 0.99
\end{align}

Where:
\begin{itemize}
    \item $R_{token}^{current} = V_{token} + (S_{bonding} - S_{sold})$
    \item $R_{amica}^{current} = V_{amica} + S_{sold} \times \frac{V_{amica}}{V_{token}}$
    \item $V_{token} = \frac{S_{bonding}}{10}$ (virtual token reserve)
    \item $V_{amica} = 100,000$ (virtual AMICA reserve)
    \item $S_{bonding}$ = bonding curve supply (222M or 333M)
    \item $S_{sold}$ = tokens already sold
\end{itemize}

\subsection{Staking Reward Calculations}

Per-block reward distribution:

\begin{align}
R_{pool} = \frac{A_{pool}}{\sum A_i} \times R_{block} \times M_{agent}
\end{align}

Where:
\begin{itemize}
    \item $A_{pool}$ = allocation points for pool
    \item $R_{block}$ = AMICA per block emission
    \item $M_{agent}$ = 1.5 if agent pool, 1.0 otherwise
\end{itemize}

User accumulated rewards:

\begin{align}
R_{user} = S_{user} \times \left(\frac{R_{accumulated}}{S_{total}} - D_{user}\right)
\end{align}

Where:
\begin{itemize}
    \item $S_{user}$ = user's staked amount
    \item $R_{accumulated}$ = total pool rewards
    \item $S_{total}$ = total staked in pool
    \item $D_{user}$ = user's reward debt
\end{itemize}

\subsection{Agent Token Distribution}

For personas with agent tokens, rewards per depositor:

\begin{align}
R_{depositor} = \frac{D_{user}}{D_{total}} \times S_{rewards}
\end{align}

Where:
\begin{itemize}
    \item $D_{user}$ = user's agent token deposits
    \item $D_{total}$ = total agent tokens deposited
    \item $S_{rewards}$ = 222,222,223 persona tokens
\end{itemize}

\section{Technical Specifications}

\subsection{Contract Interfaces}

Key functions for integration:

\begin{verbatim}
interface IPersonaTokenFactory {
    function createPersona(
        address pairingToken,
        string memory name,
        string memory symbol,
        string[] memory metadataKeys,
        string[] memory metadataValues,
        uint256 initialBuyAmount,
        address agentToken,
        uint256 minAgentTokens
    ) external returns (uint256);
    
    function swapExactTokensForTokens(
        uint256 tokenId,
        uint256 amountIn,
        uint256 amountOutMin,
        address to,
        uint256 deadline
    ) external returns (uint256);
    
    function depositAgentTokens(
        uint256 tokenId,
        uint256 amount
    ) external;
    
    function claimAgentRewards(uint256 tokenId) external;
}

interface IPersonaStakingRewards {
    function stake(uint256 poolId, uint256 amount) external;
    function withdraw(uint256 poolId, uint256 amount) external;
    function claim(uint256 poolId) external;
    function pendingRewards(uint256 poolId, address user) 
        external view returns (uint256);
}

interface IAmicaBridgeWrapper {
    function wrap(uint256 amount) external;
    function unwrap(uint256 amount) external;
}
\end{verbatim}

\subsection{Configuration Parameters}

Default protocol settings:

\begin{itemize}
    \item Mint Cost: 1,000 AMICA
    \item Graduation Threshold: 1,000,000 pairing tokens
    \item Base Trading Fee: 1\% (100 basis points)
    \item Creator Fee Share: 50\% (5,000 basis points)
    \item Min AMICA for Discount: 1,000 AMICA
    \item Max AMICA for Discount: 1,000,000 AMICA
    \item Snapshot Delay: 100 blocks
    \item Agent Pool Boost: 1.5x
    \item Virtual AMICA Reserve: 100,000
    \item Virtual Token Ratio: 10\%
\end{itemize}

\end{appendices}

\end{document}
