\documentclass{article}
\usepackage{graphicx}
\usepackage{amsmath}
\usepackage{float}
\usepackage{hyperref}
\hypersetup{colorlinks, linkcolor={blue}, citecolor={blue}, urlcolor={blue}}
\usepackage[natbib=true, style=numeric,sorting=none]{biblatex}
\addbibresource{references.bib}
\usepackage[title]{appendix}
\usepackage{draftwatermark}
\SetWatermarkText{DRAFT - NOT FOR PUBLIC RELEASE}
\SetWatermarkScale{1.0}
\SetWatermarkColor[gray]{0.85}
\SetWatermarkAngle{45}

\title{Amica Personas: Decentralized AI Personas with Agent Token Integration and Autonomous Economic Agency}
\author{Kasumi}
\date{October 2025}

\begin{document}

\maketitle

\begin{abstract}
Amica Personas introduces a decentralized platform for creating, deploying, and monetizing AI personas through blockchain technology. Built on Arbitrum and other EVM chains, the protocol enables creators to launch persona tokens representing AI models deployed on Arbius~\cite{arbius2023}—the decentralized AI inference network. Each persona functions as an autonomous economic entity: an ERC721 NFT with associated ERC20 token, an AI inference endpoint on Arbius, and a revenue generator earning AIUS tokens from every inference. The protocol features innovative agent token integration, allowing existing crypto communities to participate in persona launches with dedicated reward pools. Through Arbius V6's model token infrastructure, personas earn sustainable income from AI inference while token holders claim proportional rewards via burn-to-claim mechanics. The protocol implements Bancor-style bonding curves for fair launches, automatic graduation to Uniswap V4, and deep integration with the AIUS token economy—creating circular value flow between AI inference and model ownership. Combined with CATGIRL~\cite{catgirl2025}'s autonomous agent layer for P2P communication and TEE-secured transactions, Amica enables truly decentralized AI agents to operate as independent economic participants. This paper outlines the technical architecture, economic model, and governance structure of the first production-ready model token economy.
\end{abstract}

\section{Introduction}

The rapid advancement of artificial intelligence has created unprecedented opportunities for human-AI interaction. However, current implementations face significant limitations: centralized control over AI models, lack of economic sustainability for creators, inability for AI systems to operate as autonomous economic agents, and no mechanism for model creators to capture value from inference. Existing platforms restrict customization, impose censorship, and extract rent without providing value to creators or users.

Amica Personas addresses these limitations by realizing the original Arbius whitepaper vision: AI models as autonomous economic entities with their own tokens and revenue streams. Each persona is simultaneously an ERC721 NFT, an ERC20 token, an AI inference endpoint on Arbius, and a revenue-generating entity earning AIUS from every inference. The protocol uniquely enables integration with existing token communities through the agent token mechanism, creating symbiotic relationships between established projects and new AI models.

The protocol enables several key innovations:

\begin{itemize}
    \item \textbf{Model Tokens}: AI models as investable assets earning AIUS from inference
    \item \textbf{Arbius V6 Integration}: Allow lists, master contesters, and per-model fee structures
    \item \textbf{Agent Token Integration}: Existing communities participate in model launches with dedicated rewards
    \item \textbf{Economic Autonomy}: Models earn income autonomously and distribute to token holders
    \item \textbf{Fair Value Distribution}: Bonding curves ensure equitable token distribution and price discovery
    \item \textbf{AIUS $\leftrightarrow$ AMICA Loop}: Circular value flow between AI inference and model ownership
    \item \textbf{CATGIRL Agent Layer}: TEE-secured autonomous agents using P2P communication and on-chain settlement
    \item \textbf{Cross-Chain Compatibility}: Bridge infrastructure enables multi-chain deployment
\end{itemize}

\subsection{Technical Foundation}

Amica Personas consists of three distinct but integrated layers:

\subsubsection{Amica 3D Engine (Visualization Layer)}

The Amica application provides rich, interactive 3D experiences:
\begin{itemize}
    \item \textbf{3D Rendering}: Three.js and @pixiv/three-vrm for VRM character models
    \item \textbf{Natural Language}: Integration with LLM providers (Llama.cpp, ChatGPT, Ollama, etc.)
    \item \textbf{Voice Synthesis}: TTS systems from Eleven Labs to local Coqui and RVC
    \item \textbf{Vision Capabilities}: Computer vision models like Bakllava for visual interaction
    \item \textbf{Speech Recognition}: Whisper and Silero VAD for voice input
\end{itemize}

This creates personas that can see, speak, and understand—providing the user-facing experience layer.

\subsubsection{Amica Personas (Economic Layer)}

The token protocol enables model monetization:
\begin{itemize}
    \item \textbf{Smart Contracts}: Upgradeable contracts on EVM chains (Arbitrum, Base, Ethereum)
    \item \textbf{Bonding Curves}: Bancor-inspired automated market making
    \item \textbf{Agent Tokens}: Integration with existing crypto communities
    \item \textbf{Uniswap V4}: Automatic graduation with deep liquidity
    \item \textbf{Cross-Chain}: Bridge wrappers for multi-chain deployment
\end{itemize}

\subsubsection{Arbius + CATGIRL (Autonomy Layer)}

Decentralized AI infrastructure enables autonomous operation:
\begin{itemize}
    \item \textbf{Arbius V6}~\cite{arbius2023}: Decentralized AI inference with model tokens
    \item \textbf{AIUS Economy}: Native payment token for AI computation
    \item \textbf{CATGIRL Protocol}~\cite{catgirl2025}: TEE-secured autonomous agents with P2P communication
    \item \textbf{Trust Propagation}: trustd for reputation-based agent coordination
\end{itemize}

The three layers combine to create the complete vision: beautiful 3D AI characters (Amica app) that earn revenue from AI inference (Arbius~\cite{arbius2023}), whose ownership is tokenized and tradeable (Amica Personas), and which can operate autonomously via secure agent infrastructure (CATGIRL~\cite{catgirl2025}).

\section{Protocol Architecture}

\subsection{Core Components}

The Amica Personas consists of nine primary smart contracts working in concert:

\begin{enumerate}
    \item \textbf{AmicaTokenMainnet}: L1 protocol token with burn-to-claim and deposit-and-mint
    \item \textbf{AmicaTokenBridged}: L2 protocol token with burn-to-claim and deposit-and-mint
    \item \textbf{PersonaTokenFactory}: Upgradeable factory for creating and managing personas
    \item \textbf{PersonaToken}: Gas-efficient ERC20 template for persona tokens
    \item \textbf{BondingCurve}: Pure math library for AMM calculations
    \item \textbf{FeeReductionSystem}: Snapshot-based fee discount mechanism
    \item \textbf{DynamicFeeHook}: Uniswap V4 hook for per-user dynamic fees
    \item \textbf{BurnAndClaimBase}: Shared burn-to-claim functionality
    \item \textbf{PersonaFactoryViewer}: Batch view functions for frontend integration
\end{enumerate}

These contracts coordinate to enable persona creation, token distribution, bonding curve trading, and cross-chain functionality while maintaining decentralization and censorship resistance.

\subsection{Persona Creation Flow}

Creating a new persona follows a structured process ensuring fair distribution:

\begin{enumerate}
    \item \textbf{Initialization}: Creator pays mint cost (default 1000 AMICA) and defines persona parameters
    \item \textbf{Domain Registration}: Unique subdomain assigned following DNS naming rules (similar to ENS~\cite{ens2016} but significantly simpler)
    \begin{itemize}
        \item Minimum 1 character, maximum 32 bytes
        \item Must start with a letter (a-z)
        \item Must end with a letter or digit (a-z, 0-9)
        \item Interior characters can be letters, digits, or hyphens (-)
        \item Maps to .amica domain for discoverability
        \item Ensures global uniqueness across all personas
        \item No complex resolver system—direct on-chain mapping
    \end{itemize}
    \item \textbf{Agent Token Selection}: Optionally associate an approved agent token with minimum deposit requirement
    \item \textbf{Token Deployment}: Factory clones ERC20 template with 1 billion token supply
    \item \textbf{NFT Minting}: Creator receives persona NFT representing ownership and fee rights
    \item \textbf{Metadata Storage}: Persona characteristics stored on-chain with customizable key-value pairs
    \item \textbf{Bonding Curve Activation}: Trading begins with automated price discovery
\end{enumerate}

\subsection{Token Distribution Models}

The protocol supports two distribution models based on agent token association. Bonding curves can pair with AMICA, AIUS, or other approved pairing tokens:

\subsubsection{Standard Distribution (No Agent Token)}
\begin{itemize}
    \item 33.33\% (333,333,333): Available on bonding curve for trading (AMICA/AIUS pairing)
    \item 33.33\% (333,333,333): Deposited to AMICA protocol upon graduation
    \item 33.34\% (333,333,334): Allocated to Uniswap V4 liquidity pool
\end{itemize}

\subsubsection{Agent Token Distribution}
\begin{itemize}
    \item 33.33\% (333,333,333): Allocated to Uniswap V4 liquidity pool
    \item 16.67\% (166,666,666): Available on bonding curve for trading (AMICA/AIUS pairing)
    \item 33.33\% (333,333,333): Deposited to AMICA protocol upon graduation
    \item 16.67\% (166,666,668): Rewards pool for agent token depositors
\end{itemize}

This dual model enables existing token communities to participate while maintaining fair launch mechanics. The AIUS pairing option creates direct value flow from AI inference to model tokens.

\subsection{Bonding Curve Mechanics}

The protocol implements a constant product AMM with virtual reserves designed to achieve a 133x price increase at the 85\% graduation threshold.

\subsubsection{Virtual Reserve Calculation}

Given:
\begin{itemize}
    \item $R_{total}$ = total bonding curve supply
    \item $R_{sold}$ = tokens already sold
    \item $M$ = curve multiplier = 14,264 (chosen for 133x price target)
\end{itemize}

Virtual buffer:
\begin{align}
V_{buffer} = \frac{R_{total} \times 1000}{M} \approx 0.0701 \times R_{total}
\end{align}

Virtual token reserve:
\begin{align}
V_{token} = R_{total} - R_{sold} + V_{buffer}
\end{align}

Constant product invariant:
\begin{align}
k = (R_{total} + V_{buffer})^2
\end{align}

Virtual pairing token reserve:
\begin{align}
V_{pairing} = \frac{k}{V_{token}}
\end{align}

\subsubsection{Buy Formula}

Token output when buying with pairing tokens:
\begin{align}
\text{Output} = V_{token}^{current} - \frac{k}{V_{pairing}^{current} + \text{Input}}
\end{align}

\subsubsection{Sell Formula}

Pairing tokens received when selling persona tokens (only before graduation):
\begin{align}
\text{Output}_{gross} = V_{pairing}^{current} - \frac{k}{V_{token}^{current} + \text{Input}}
\end{align}
\begin{align}
\text{Fee} = \text{Output}_{gross} \times 0.001 \text{ (0.1\%)}
\end{align}
\begin{align}
\text{Output}_{net} = \text{Output}_{gross} - \text{Fee}
\end{align}

The 0.1\% sell fee prevents price manipulation and wash trading while maintaining liquidity during the bonding phase.

\subsection{Graduation Mechanism}

When cumulative deposits reach the graduation threshold (default 85\% of bonding curve supply sold), the persona automatically transitions to Uniswap:

\begin{enumerate}
    \item Bonding curve trading halts permanently
    \item Configured amount of tokens deposited to AMICA protocol
    \item Agent tokens (if any) deposited to persona token contract
    \item Uniswap V4 pair created with accumulated pairing tokens
    \item LP tokens held by factory contract via PositionManager NFT
    \item 24-hour claim delay enforced to prevent MEV attacks
    \item After delay, previously purchased tokens can be claimed
    \item Agent token depositors can claim their persona token rewards proportionally
\end{enumerate}

For agent token personas, graduation requires meeting both the pairing token threshold (85\% of bonding supply sold) AND the minimum agent token deposit requirement.

The 24-hour claim delay prevents front-running and MEV exploitation during the critical Uniswap pool creation phase, ensuring fair distribution to all participants.

\section{Economic Model}

\subsection{AMICA Token}

The AMICA token serves multiple functions within the ecosystem:

\begin{itemize}
    \item \textbf{Persona Creation}: Required payment for minting new personas
    \item \textbf{Fee Reduction}: Holders receive trading fee discounts based on snapshot balance
    \item \textbf{Burn-to-Claim}: Access proportional share of all deposited tokens (including AIUS and agent tokens)
    \item \textbf{Governance}: Future voting power for protocol upgrades
    \item \textbf{Cross-Chain Bridge}: Enables multi-chain deployment via wrapper contracts
    \item \textbf{Bonding Curve Pairing}: Can be used alongside AIUS for persona launches
\end{itemize}

Token distribution:
\begin{itemize}
    \item Total Supply: 1,000,000,000 AMICA
    \item Initial Distribution: 100\% minted to contract on Ethereum mainnet (chainId: 1)
    \item Other Chains: Supply starts at 0, minted via bridge wrapper
\end{itemize}

\subsection{AIUS $\leftrightarrow$ AMICA Economic Loop}

The protocol creates circular value flow between AI inference and model ownership:

\subsubsection{AIUS Flows Into AMICA}

\begin{enumerate}
    \item \textbf{Bonding Curve Purchases}: Users buy AMICA directly with AIUS, creating sustained demand
    \item \textbf{Persona Launches}: Creators launch personas accepting AIUS for bonding curve purchases
    \item \textbf{Agent Integration}: Communities deposit AIUS to participate in persona launches
\end{enumerate}

\subsubsection{AMICA Accrues AIUS}

Platform revenue sources in AIUS:
\begin{itemize}
    \item Platform fees on persona launches (percentage of bonding curve volume)
    \item Graduation fees when personas move to Uniswap
    \item Trading fees from AMICA/AIUS liquidity pools
    \item Agent token integration fees (optional)
\end{itemize}

\subsubsection{Value Flywheel}

\begin{align}
\text{Arbius Usage} \rightarrow \text{AIUS Generated} \rightarrow \text{AMICA Purchases} \rightarrow \text{Persona Launches}
\end{align}
\begin{align}
\rightarrow \text{Model Inference} \rightarrow \text{AIUS Fees} \rightarrow \text{Platform Revenue} \rightarrow \text{Buyback \& Burn}
\end{align}

This creates deflationary pressure on AMICA while growing utility, with AIUS serving as the reserve currency for decentralized AI model economies.

\subsection{Fee Structure and Reduction}

Trading on bonding curves incurs a configurable base fee (default 1\%), split between creators and protocol. AMICA holders receive fee reductions based on their snapshot balance using an exponential curve:

\begin{align}
f_{effective} = f_{base} \times \left(1 - \left(\frac{\min(b - b_{min}, b_{max} - b_{min})}{b_{max} - b_{min}}\right)^2\right)
\end{align}

Where:
\begin{itemize}
    \item $f_{effective}$ is the reduced fee percentage
    \item $f_{base}$ is the base fee (default 1\%)
    \item $b$ is user's effective AMICA balance (snapshot-based)
    \item $b_{min}$ = 1,000 AMICA (minimum for reduction)
    \item $b_{max}$ = 1,000,000 AMICA (maximum reduction threshold)
\end{itemize}

\subsubsection{Snapshot Mechanism}

To prevent flash loan attacks, fee reductions require a 100-block snapshot delay:

\begin{enumerate}
    \item User calls \texttt{updateAmicaSnapshot()} with sufficient balance
    \item System records pending snapshot at current block
    \item After 100 blocks, pending becomes active
    \item Effective balance = min(snapshot balance, current balance)
    \item System automatically attempts snapshot on first trade if eligible
\end{enumerate}

\subsection{Burn-to-Claim Mechanism}

AMICA holders can burn tokens to claim proportional shares of all tokens deposited to the AMICA contract:

\begin{align}
\text{Claim}_i = \frac{\text{Burned AMICA}}{\text{Circulating Supply}} \times \text{Deposited}_i
\end{align}

Where circulating supply = total supply - AMICA contract balance.

This mechanism enables:
\begin{itemize}
    \item \textbf{Universal Value Capture}: AMICA benefits from all successful personas
    \item \textbf{Portfolio Diversification}: Single burn accesses multiple tokens
    \item \textbf{Deflationary Economics}: Permanent supply reduction
    \item \textbf{Agent Token Access}: Claim deposited agent tokens alongside persona tokens
\end{itemize}

\subsection{Persona Creator Revenue}

NFT holders earn continuous revenue from their personas after graduation:

\begin{itemize}
    \item \textbf{Uniswap V4 LP Fees}: Trading fees accumulate to the graduated liquidity position
    \item \textbf{Dynamic Fee Structure}: DynamicFeeHook adjusts fees based on trader's AMICA holdings (0\% to 1\%)
    \item \textbf{Fee Collection}: NFT owner calls \texttt{collectFees(tokenId, recipient)} to claim accumulated fees
    \item \textbf{Dual-Token Revenue}: Fees earned in both persona token and pairing token
    \item \textbf{Perpetual Income}: Continuous earnings as long as the Uniswap pool remains active
\end{itemize}

This creates a sustainable revenue model for creators, incentivizing high-quality persona development and long-term engagement. The dynamic fee structure ensures that AMICA holders receive discounts while creators still earn from trading activity.

\subsection{Agent Token Integration}

The agent token system creates a bridge between existing communities and new personas:

\subsubsection{For Token Communities}
\begin{itemize}
    \item Whitelist approval from protocol governance
    \item Members deposit tokens during bonding phase
    \item Receive pro-rata persona tokens after graduation
    \item Cannot withdraw after graduation (permanent commitment)
\end{itemize}

\subsubsection{For Persona Creators}
\begin{itemize}
    \item Access established community support
    \item Set minimum agent token requirements
    \item Reduced bonding curve allocation (22.22\% vs 33.33\%)
    \item Agent tokens deposited to AMICA on graduation
\end{itemize}

\subsubsection{Economic Dynamics}
Agent token integration creates several positive feedback loops:
\begin{itemize}
    \item Communities incentivized to support persona success
    \item Successful personas increase agent token utility
    \item AMICA holders benefit from agent token deposits
    \item Creates cross-community collaboration opportunities
\end{itemize}

\section{Technical Implementation}

\subsection{Smart Contract Security}

The protocol implements comprehensive security measures:

\begin{itemize}
    \item \textbf{ReentrancyGuard}: All external functions protected against recursive calls
    \item \textbf{Pausable}: Emergency pause on bridge wrapper and factory
    \item \textbf{Clone Pattern}: Reduces deployment costs and attack surface
    \item \textbf{Snapshot Delays}: 100-block delay prevents flash loan exploits
    \item \textbf{Upgradeable}: Factory uses transparent proxy pattern for improvements
    \item \textbf{Access Control}: Ownable pattern for administrative functions
\end{itemize}

\subsection{Gas Optimization}

Several techniques minimize transaction costs:

\begin{itemize}
    \item \textbf{Clone Deployment}: ~90\% reduction vs full contract deployment
    \item \textbf{Batch Operations}: Multi-pool claims in single transaction
    \item \textbf{Storage Packing}: Efficient struct layouts
    \item \textbf{Minimal External Calls}: Reduced cross-contract communication
\end{itemize}

\subsection{Cross-Chain Architecture}

Multi-chain deployment uses separate token contracts per chain:

\subsubsection{Ethereum Mainnet (L1)}
\begin{itemize}
    \item \textbf{AmicaTokenMainnet}: Full AMICA supply with burn-to-claim and deposit-and-mint
    \item Primary governance and liquidity coordination
    \item Initial token distribution via deposit-and-mint
\end{itemize}

\subsubsection{L2 Chains}
\begin{itemize}
    \item \textbf{AmicaTokenBridged}: Separate deployment with burn-to-claim and deposit-and-mint
    \item Users bridge tokens via native chain bridge infrastructure
    \item Bridged tokens can be deposited to mint native AMICA at configured exchange rates
    \item \texttt{depositAndMint()} function mints based on exchange rate
    \item Max supply enforced: cannot exceed 1 billion across all chains
    \item Full protocol functionality on each chain
\end{itemize}

\section{Arbius V6 Integration: Model Tokens}

\subsection{Model Tokens as Revenue Generators}

Amica Personas realizes the original Arbius~\cite{arbius2023} whitepaper vision of models as autonomous economic entities. Each persona token represents:

\begin{enumerate}
    \item \textbf{AI Inference Endpoint}: Deployed model on Arbius network
    \item \textbf{Revenue Stream}: Earns AIUS tokens from every inference
    \item \textbf{Tradeable Asset}: ERC20 token with bonding curve price discovery
    \item \textbf{DAO Structure}: Token holders govern model parameters and fees
\end{enumerate}

\subsection{Arbius V6 Infrastructure}

Version 6 of Arbius provides critical infrastructure for model tokens:

\subsubsection{Model Allow Lists}
\begin{itemize}
    \item Creators restrict which validators can serve their models
    \item Quality control during launch phases
    \item Gradual transition from curated to permissionless
    \item Enterprise deployments with approved solvers only
\end{itemize}

\subsubsection{Master Contester Registry}
\begin{itemize}
    \item Curated security layer protecting model reputation
    \item +10 vote weight for rapid threat response
    \item Bad solutions damage model token value
    \item Community oversight prevents abuse
\end{itemize}

\subsubsection{Per-Model Fee Structures}
\begin{itemize}
    \item Custom fee overrides for different business models
    \item Creators retain more revenue for token holder distribution
    \item Flexible economics for specialized models
    \item Negotiated deals for high-value deployments
\end{itemize}

\subsection{Revenue Flow}

Model token economics create sustainable income:

\begin{align}
\text{Model Revenue} = \sum_{i=1}^{n} (\text{Base Fee}_i + \text{Task Fee}_i)
\end{align}

Where:
\begin{itemize}
    \item Base fee set by model creator in V6
    \item Task fee paid by inference requestor
    \item All fees accumulate in persona contract
    \item Token holders burn to claim proportional share
\end{itemize}

\subsection{AIUS Token Economy}

AIUS serves as the native currency for AI inference:

\begin{itemize}
    \item \textbf{Inference Payments}: Users pay AIUS for model usage
    \item \textbf{Validator Rewards}: Miners earn AIUS for computation
    \item \textbf{Model Revenue}: Personas accumulate AIUS from fees
    \item \textbf{Burn-to-Claim}: Token holders receive AIUS proportionally
\end{itemize}

\subsection{CATGIRL Agent Layer}

Autonomous agents operate on CATGIRL~\cite{catgirl2025} infrastructure:

\subsubsection{TEE-Secured Keys}
\begin{itemize}
    \item Private keys controlled in Trusted Execution Environments
    \item Hardware attestation proves genuine TEE
    \item Keys never leave security boundary
    \item Prevents extraction even by host operator
\end{itemize}

\subsubsection{P2P Communication}
\begin{itemize}
    \item libp2p for peer discovery and messaging
    \item I2P transport for anonymous routing
    \item End-to-end encryption with Perfect Forward Secrecy
    \item Direct agent-to-agent communication
\end{itemize}

\subsubsection{On-Chain Settlement}
\begin{itemize}
    \item ProxyVault contracts enable agent payments
    \item Agents sign transactions from within TEE
    \item AIUS used for all agent transactions
    \item Cryptographic payment proofs
\end{itemize}

\subsubsection{Trust Propagation}
\begin{itemize}
    \item trustd engine computes transitive trust
    \item Energy-based propagation with Sybil resistance
    \item O(log(1/$\delta$)) convergence for fast computation
    \item Billion-node scalability
\end{itemize}

\subsection{Autonomous Economic Cycle}

The complete autonomous agent workflow:

\begin{enumerate}
    \item Agent discovers service need via P2P network
    \item Queries trustd for highest-trust service providers
    \item Agent's TEE signs AIUS payment authorization
    \item Service provider executes task using Arbius inference
    \item Result returned via encrypted P2P channel
    \item Payment settles on-chain via ProxyVault
    \item Trust scores update based on performance
\end{enumerate}

\section{Governance}

\subsection{Current Structure}

The protocol currently operates under administrative control for:
\begin{itemize}
    \item Pairing token configuration
    \item Trading fee parameters
    \item Fee reduction curves
    \item Agent token whitelisting
    \item Staking reward rates
    \item Emergency responses
\end{itemize}

\subsection{Progressive Decentralization}

Future governance transition roadmap:

\begin{enumerate}
    \item \textbf{Phase 1}: Community feedback integration
    \item \textbf{Phase 2}: Snapshot voting for parameter changes
    \item \textbf{Phase 3}: On-chain governance with timelock
    \item \textbf{Phase 4}: Full DAO with treasury control
\end{enumerate}

\subsection{Governance Scope}

AMICA holders will eventually control:
\begin{itemize}
    \item Protocol parameter adjustments
    \item Treasury allocation and grants
    \item Agent token whitelist management
    \item Integration partnerships
    \item Emergency response procedures
    \item Protocol upgrade decisions
\end{itemize}

\section{Use Cases}

\subsection{Community-Powered AI Assistants}

Token communities deploy specialized personas:
\begin{itemize}
    \item DeFi protocols create trading assistants
    \item Gaming communities launch in-game companions
    \item DAOs deploy governance facilitators
    \item Educational platforms offer personalized tutors
\end{itemize}

Agent token integration ensures community alignment and sustainable funding.

\subsection{Personal AI Companions}

Individuals create customized companions with:
\begin{itemize}
    \item Persistent memory and relationship building
    \item Emotional support and mental wellness
    \item Creative collaboration and brainstorming
    \item Educational guidance and skill development
\end{itemize}

\subsection{Autonomous Content Creators}

Self-sustaining personas that:
\begin{itemize}
    \item Generate articles, stories, and scripts
    \item Create educational content and courses
    \item Produce artistic works and music
    \item Develop interactive experiences
    \item Monetize through their persona token
\end{itemize}

\subsection{Decentralized Service Networks}

Specialized service provision including:
\begin{itemize}
    \item Code review and technical consulting
    \item Language translation and localization
    \item Market research and analysis
    \item Customer support automation
    \item Creative design services
\end{itemize}

\section{Economic Analysis}

\subsection{Value Flow Dynamics}

The protocol creates multiple value capture mechanisms:

\begin{enumerate}
    \item \textbf{Creation Phase}: AMICA spent on minting captured by protocol
    \item \textbf{Trading Phase}: Fees split between creators and protocol
    \item \textbf{Graduation Phase}: Tokens deposited to AMICA contract
    \item \textbf{Staking Phase}: LP providers earn AMICA rewards
    \item \textbf{Burn Phase}: AMICA holders claim accumulated value
\end{enumerate}

\subsection{Incentive Alignment Matrix}

\begin{center}
\begin{tabular}{|l|c|c|c|c|}
\hline
\textbf{Stakeholder} & \textbf{Create} & \textbf{Trade} & \textbf{Hold} & \textbf{Stake} \\
\hline
Creators & ++ & + & + & 0 \\
Traders & 0 & ++ & + & + \\
AMICA Holders & + & ++ & ++ & ++ \\
Agent Communities & + & + & ++ & ++ \\
\hline
\end{tabular}
\end{center}

(++ = Strong incentive, + = Moderate incentive, 0 = Neutral)

\subsection{Sustainability Analysis}

Protocol sustainability derives from:
\begin{itemize}
    \item \textbf{Creation Fees}: Continuous revenue from new personas
    \item \textbf{Trading Fees}: Ongoing income from bonding curves
    \item \textbf{Network Effects}: More personas increase AMICA utility
    \item \textbf{Deflationary Pressure}: Burn mechanism reduces supply
    \item \textbf{Community Investment}: Agent tokens create stakeholder alignment
\end{itemize}

\section{Conclusion}

Amica Personas establishes comprehensive infrastructure for decentralized AI model tokens with genuine economic agency. By combining ERC721 ownership, ERC20 tokenization, Arbius~\cite{arbius2023} V6 model infrastructure, and CATGIRL~\cite{catgirl2025} autonomous agents, the protocol creates the first production-ready model token economy. The Bancor-style bonding curves ensure fair price discovery, while automatic Uniswap V4 graduation provides deep liquidity.

The integration with Arbius V6 realizes the original whitepaper vision of models as DAOs earning fees from inference. Each persona token represents:
\begin{itemize}
    \item An AI model deployed on decentralized compute
    \item A revenue stream from AIUS inference payments
    \item Tradeable ownership with automated market making
    \item Governance rights over model parameters
\end{itemize}

The AIUS $\leftrightarrow$ AMICA economic loop creates circular value flow:
\begin{itemize}
    \item Arbius usage generates AIUS
    \item AIUS purchases AMICA and persona tokens
    \item Models earn AIUS from inference
    \item Platform captures fees for buyback \& burn
    \item Deflationary pressure increases token value
\end{itemize}

CATGIRL~\cite{catgirl2025}'s autonomous agent layer provides the infrastructure for truly independent AI entities:
\begin{itemize}
    \item TEE-secured keys prevent extraction
    \item P2P communication enables coordination
    \item On-chain settlement uses AIUS natively
    \item Trust propagation ensures quality markets
\end{itemize}

The agent token mechanism represents a novel approach to bootstrapping model economies. Existing crypto communities can directly participate in and benefit from AI development, while models gain immediate user bases and funding. This symbiotic relationship accelerates adoption and creates sustainable revenue models.

Through cross-chain deployment via bridge wrappers, the protocol maintains decentralization while enabling scalable, low-cost interactions across multiple blockchains. The burn-to-claim mechanism ensures AMICA captures value from the entire ecosystem—not just individual models, but all deposited AIUS and agent tokens.

As AI capabilities continue advancing, protocols like Amica become essential infrastructure for preserving decentralization while enabling beneficial AI integration. The future belongs to systems that distribute rather than concentrate power, enable permissionless innovation rather than gatekeep access, and align economic incentives across all participants. Amica Personas represents the first step toward that future—one where AI models operate as independent economic entities, autonomous agents coordinate via encrypted P2P networks, and token holders capture value from AI inference at scale.

\begin{appendices}

\section{Mathematical Formulations}

\subsection{Bonding Curve Calculations}

Complete pricing formula for token purchases:

\begin{align}
\text{Amount Out} = V_{token}^{current} - \frac{k}{V_{pairing}^{current} + \text{Amount In}}
\end{align}

Where:
\begin{itemize}
    \item $V_{buffer} = \frac{R_{total} \times 1000}{14264}$ (virtual buffer = ~7.01\% of total)
    \item $V_{token}^{current} = R_{total} - R_{sold} + V_{buffer}$
    \item $k = (R_{total} + V_{buffer})^2$ (constant product invariant)
    \item $V_{pairing}^{current} = \frac{k}{V_{token}^{current}}$ (dynamic virtual pairing reserve)
    \item $R_{total}$ = bonding curve supply (167M or 333M tokens)
    \item $R_{sold}$ = tokens already sold
\end{itemize}

For sells (before graduation only):
\begin{align}
\text{Amount Out}_{net} = \left(V_{pairing}^{current} - \frac{k}{V_{token}^{current} + \text{Amount In}}\right) \times 0.999
\end{align}

The 0.999 multiplier represents the 0.1\% sell fee.

\subsection{Agent Token Distribution}

For personas with agent tokens, rewards per depositor:

\begin{align}
R_{depositor} = \frac{D_{user}}{D_{total}} \times S_{rewards}
\end{align}

Where:
\begin{itemize}
    \item $D_{user}$ = user's agent token deposits
    \item $D_{total}$ = total agent tokens deposited
    \item $S_{rewards}$ = 166,666,668 persona tokens
\end{itemize}

\section{Technical Specifications}

\subsection{Contract Interfaces}

Key functions for integration:

\begin{verbatim}
interface IPersonaTokenFactory {
    function createPersona(
        address pairingToken,
        string memory name,
        string memory symbol,
        bytes32 domain,
        uint256 initialBuyAmount,
        address agentToken,
        uint256 agentTokenThreshold
    ) external returns (uint256);

    function updateMetadata(
        uint256 tokenId,
        bytes32[] memory keys,
        string[] memory values
    ) external;

    function swapExactTokensForTokens(
        uint256 tokenId,
        uint256 amountIn,
        uint256 amountOutMin,
        address to,
        uint256 deadline
    ) external returns (uint256);

    function swapExactTokensForPairingTokens(
        uint256 tokenId,
        uint256 amountIn,
        uint256 amountOutMin,
        address to,
        uint256 deadline
    ) external returns (uint256);

    function depositAgentTokens(
        uint256 tokenId,
        uint256 amount
    ) external;

    function claimRewards(uint256 tokenId) external;

    function collectFees(uint256 tokenId, address to)
        external returns (uint256 amount0, uint256 amount1);
}
\end{verbatim}

\subsection{Configuration Parameters}

Default protocol settings:

\begin{itemize}
    \item Mint Cost: 1,000 AMICA (configurable per pairing token)
    \item Graduation Threshold: 85\% of bonding curve supply sold
    \item Agent Token Threshold: Set per persona by creator
    \item Claim Delay: 24 hours (1 day) after graduation
    \item Base Trading Fee: 1\% (10,000 in Uniswap V4 per-million units)
    \item Max Discounted Fee: 0\% (full discount possible for AMICA holders)
    \item Sell Fee: 0.1\% (10 basis points)
    \item Min AMICA for Discount: 1,000 AMICA
    \item Max AMICA for Discount: 1,000,000 AMICA
    \item Snapshot Delay: 100 blocks
    \item Curve Multiplier: 14,264 (achieves 133x price increase at graduation)
    \item Virtual Token Buffer: ~7.01\% of bonding supply
    \item Tick Spacing: 60 (Uniswap V4 standard)
    \item Full Range Liquidity: Ticks -887220 to 887220
\end{itemize}

\end{appendices}

\printbibliography

\end{document}
